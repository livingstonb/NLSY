\documentclass{article}

\usepackage[margin=1in]{geometry}
\usepackage{pdflscape}
\usepackage{graphicx}
\usepackage{subcaption}

\begin{document}

\tableofcontents
\newpage

\section{Sample description}
I use non-black, non-hispanic men and women in the NLSY97 interviews. I define adult height as the height reported in the last available year, 2011. I use wage per hour in 2014, computed as annual wage and salary income divided by hours worked. Annual wage and salary income is topcoded at the top 2\% of income. I exclude from the sample individuals with less than 1000 hours worked in 2014 and individuals with an hourly wage of less than \$1. 
For completed years of schooling, I use variables indicating the highest grade completed, including years of college. This definition differs slightly from the definition used in Persico et al (2004), who use the age at which an individual left school - 5. Residential parents (as opposed to biological) were used for the parents' highest grade completed.

ASVAB scores are the respondent's scores on the Armed Services Vocational Aptitude Battery (CAT-ASVAB), administered in 1997-1998. This test measures the respondent's knowledge and skills in 12 different topics.

Unfortunately NLSY97 does not have appear to have data on self esteem, parent occupation, or high school activities. We do have data on household income and net worth in 1997, as well as some variables on the home and neighborhood environment.

\section{Summary statistics}
Here I compute summary statistics for men and women separately, where each statistic is based on the restricted sample in which all variables in the table are nonmissing. Note the small sample sizes. I weight these statistics using the cumulative cases weights, which are designed for use when all time periods are needed. I don't think these are the correct weights but it's unclear, we'll have to discuss weighting at some point. NLSY97 has an online custom weighting program that allows the user to select specific years to produce a weighting scheme when only those years are used. It sounds like that's the way to go but I'm unsure.

\begin{table}[h]
\label{tab:summary_men}
\caption{Summary statistics for men.}
\input{../stats/output/summary_men.tex}
\end{table}

\begin{table}[h]
\label{tab:summary_women}
\caption{Summary statistics for women.}
\input{../stats/output/summary_women.tex}
\end{table}

\clearpage

% SCATTER PLOTS
\section{Scatter plots of income and height}
% BASELINE
\subsection{Baseline}
\begin{figure}[htbp]
	\begin{subfigure}[b]{0.9\textwidth}
		\centering
		\includegraphics[width=0.7\textwidth]{../stats//output/figs/loghourlywage_height_linear.png}	
		\label{fig:loghourlywageheight}
		\caption{Log hourly income vs. adult height}
	\end{subfigure}
	
	\begin{subfigure}[b]{0.9\textwidth}
		\centering
		\includegraphics[width=0.7\textwidth]{../stats//output/figs/logannualwage_height_linear.png}	
		\label{fig:logannualwageheight}
		\caption{Log annual income vs. adult height}
	\end{subfigure}
	\caption{Log hourly and annual income scatter plots. Confidence intervals computed with robust standard errors.}
\end{figure}

\clearpage
\subsection{Baseline, narrowed y-axis}
\begin{figure}[htbp]
	\begin{subfigure}[b]{0.9\textwidth}
		\centering
		\includegraphics[width=0.7\textwidth]{../stats//output/figs/loghourlywage_height_narrow_linear.png}	
		\label{fig:loghourlywageheight}
		\caption{Log hourly income vs. adult height}
	\end{subfigure}
	
	\begin{subfigure}[b]{0.9\textwidth}
		\centering
		\includegraphics[width=0.7\textwidth]{../stats//output/figs/logannualwage_height_narrow_linear.png}	
		\label{fig:logannualwageheight}
		\caption{Log annual income vs. adult height}
	\end{subfigure}
	\caption{Log hourly and annual income scatter plots.}
\end{figure}

\clearpage
\subsection{Baseline, quadratic fit}
\begin{figure}[htbp]
	\begin{subfigure}[b]{0.9\textwidth}
		\centering
		\includegraphics[width=0.7\textwidth]{../stats//output/figs/loghourlywage_height_quadr.png}	
		\label{fig:loghourlywageheight}
		\caption{Log hourly income vs. adult height}
	\end{subfigure}
	
	\begin{subfigure}[b]{0.9\textwidth}
		\centering
		\includegraphics[width=0.7\textwidth]{../stats//output/figs/logannualwage_height_quadr.png}	
		\label{fig:logannualwageheight}
		\caption{Log annual income vs. adult height}
	\end{subfigure}
	\caption{Log hourly and annual income scatter plots. Confidence intervals computed with robust standard errors.}
\end{figure}

\clearpage
\subsection{Baseline, fractional polynomial fit}
\begin{figure}[htbp]
	\begin{subfigure}[b]{0.9\textwidth}
		\centering
		\includegraphics[width=0.7\textwidth]{../stats//output/figs/loghourlywage_height_fpoly.png}	
		\label{fig:loghourlywageheight}
		\caption{Log hourly income vs. adult height}
	\end{subfigure}
	
	\begin{subfigure}[b]{0.9\textwidth}
		\centering
		\includegraphics[width=0.7\textwidth]{../stats//output/figs/logannualwage_height_fpoly.png}	
		\label{fig:logannualwageheight}
		\caption{Log annual income vs. adult height}
	\end{subfigure}
	\caption{Log hourly and annual income scatter plots. Confidence intervals computed with robust standard errors.}
\end{figure}

\clearpage
\subsection{Sample of 2014 federal minimum wage earners and above}
\begin{figure}[htbp]
	\begin{subfigure}[b]{0.9\textwidth}
		\centering
		\includegraphics[width=0.7\textwidth]{../stats//output/figs/loghourlywage_height_minwage_linear.png}	
		\label{fig:loghourlywageheight_minwage}
		\caption{Log hourly income vs. adult height}
	\end{subfigure}
	
	\begin{subfigure}[b]{0.9\textwidth}
		\centering
		\includegraphics[width=0.7\textwidth]{../stats//output/figs/logannualwage_height_minwage_linear.png}	
		\label{fig:logannualwageheight_minwage}
		\caption{Log annual income vs. adult height}
	\end{subfigure}
	\caption{Log hourly and annual income scatter plots, restricted to federal minimum wage and above. Confidence intervals computed on this subsample with robust standard errors.}
\end{figure}



\clearpage

% REGRESSION TABLES
% REGRESSIONS WITH FAMILY CONTROLS, CONTINUOUS HEIGHT
\newgeometry{margin=2cm}
\begin{landscape}
\section{Regressions with family controls, continuous height variable}
\subsection{Controlling for teen height, men}
\input{../stats/output/regressions/OLSfamily_th_hc_men.tex}
\end{landscape}

\newgeometry{margin=2cm}
\begin{landscape}
\subsection{Controlling for teen height, women}
\input{../stats/output/regressions/OLSfamily_th_hc_women.tex}
\end{landscape}

\newgeometry{margin=2cm}
\begin{landscape}
\subsection{Controlling for teen height, pooled}
\input{../stats/output/regressions/OLSfamily_th_hc_pooled.tex}
\end{landscape}

\newgeometry{margin=2cm}
\begin{landscape}
\subsection{Controlling for teen height, pooled with height-gender interaction}
\input{../stats/output/regressions/OLSfamily_th_hc_pooledinteraction.tex}
\end{landscape}

\newgeometry{margin=2cm}
\begin{landscape}
\subsection{Not controlling for teen height, men}
\input{../stats/output/regressions/OLSfamily_noth_hc_men.tex}
\end{landscape}

\newgeometry{margin=2cm}
\begin{landscape}
\subsection{Not controlling for teen height, women}
\input{../stats/output/regressions/OLSfamily_noth_hc_women.tex}
\end{landscape}

\newgeometry{margin=2cm}
\begin{landscape}
\subsection{Not controlling for teen height, pooled}
\input{../stats/output/regressions/OLSfamily_noth_hc_pooled.tex}
\end{landscape}

\newgeometry{margin=2cm}
\begin{landscape}
\subsection{Not controlling for teen height, pooled with height-gender interaction}
\input{../stats/output/regressions/OLSfamily_noth_hc_pooledinteraction.tex}
\end{landscape}

% REGRESSIONS WITH FAMILY CONTROLS, HEIGHT QUARTILES

\newgeometry{margin=2cm}
\begin{landscape}
\section{Regressions with family controls, height quartiles}
\subsection{Controlling for teen height, men}
\input{../stats/output/regressions/OLSfamily_th_hq_men.tex}
\end{landscape}

\newgeometry{margin=2cm}
\begin{landscape}
\subsection{Controlling for teen height, women}
\input{../stats/output/regressions/OLSfamily_th_hq_women.tex}
\end{landscape}

\newgeometry{margin=1.2cm}
\begin{landscape}
\subsection{Controlling for teen height, pooled}
\input{../stats/output/regressions/OLSfamily_th_hq_pooled.tex}
\end{landscape}

\newgeometry{margin=2cm}
\begin{landscape}
\subsection{Not controlling for teen height, men}
\input{../stats/output/regressions/OLSfamily_noth_hq_men.tex}
\end{landscape}

\newgeometry{margin=2cm}
\begin{landscape}
\subsection{Not controlling for teen height, women}
\input{../stats/output/regressions/OLSfamily_noth_hq_women.tex}
\end{landscape}

\newgeometry{margin=2cm}
\begin{landscape}
\subsection{Not controlling for teen height, pooled}
\input{../stats/output/regressions/OLSfamily_noth_hq_pooled.tex}
\end{landscape}

% REGRESSIONS WITH HEALTH CONTROLS, CONTINUOUS HEIGHT

\newgeometry{margin=2cm}
\begin{landscape}
\section{Regressions with health controls, continuous height variable}
\subsection{Controlling for teen height, men}
\input{../stats/output/regressions/OLShealth_th_hc_men.tex}
\end{landscape}

\newgeometry{margin=2cm}
\begin{landscape}
\subsection{Controlling for teen height, women}
\input{../stats/output/regressions/OLShealth_th_hc_women.tex}
\end{landscape}

\newgeometry{margin=2cm}
\begin{landscape}
\subsection{Controlling for teen height, pooled}
\input{../stats/output/regressions/OLShealth_th_hc_pooled.tex}
\end{landscape}

\newgeometry{margin=2cm}
\begin{landscape}
\subsection{Controlling for teen height, pooled with height-gender interaction}
\input{../stats/output/regressions/OLShealth_th_hc_pooledinteraction.tex}
\end{landscape}

\newgeometry{margin=2cm}
\begin{landscape}
\subsection{Not controlling for teen height, men}
\input{../stats/output/regressions/OLShealth_noth_hc_men.tex}
\end{landscape}

\newgeometry{margin=2cm}
\begin{landscape}
\subsection{Not controlling for teen height, women}
\input{../stats/output/regressions/OLShealth_noth_hc_women.tex}
\end{landscape}

\newgeometry{margin=2cm}
\begin{landscape}
\subsection{Not controlling for teen height, pooled}
\input{../stats/output/regressions/OLShealth_noth_hc_pooled.tex}
\end{landscape}

\newgeometry{margin=2cm}
\begin{landscape}
\subsection{Not controlling for teen height, pooled with height-gender interaction}
\input{../stats/output/regressions/OLShealth_noth_hc_pooledinteraction.tex}
\end{landscape}

% REGRESSIONS WITH HEALTH CONTROLS, HEIGHT QUARTILES

\newgeometry{margin=2cm}
\begin{landscape}
\section{Regressions with health controls, height quartiles}
\subsection{Controlling for teen height, men}
\input{../stats/output/regressions/OLShealth_th_hq_men.tex}
\end{landscape}

\newgeometry{margin=2cm}
\begin{landscape}
\subsection{Controlling for teen height, women}
\input{../stats/output/regressions/OLShealth_th_hq_women.tex}
\end{landscape}

\newgeometry{margin=1.2cm}
\begin{landscape}
\subsection{Controlling for teen height, pooled}
\input{../stats/output/regressions/OLShealth_th_hq_pooled.tex}
\end{landscape}

\newgeometry{margin=2cm}
\begin{landscape}
\subsection{Not controlling for teen height, men}
\input{../stats/output/regressions/OLShealth_noth_hq_men.tex}
\end{landscape}

\newgeometry{margin=2cm}
\begin{landscape}
\subsection{Not controlling for teen height, women}
\input{../stats/output/regressions/OLShealth_noth_hq_women.tex}
\end{landscape}

\newgeometry{margin=2cm}
\begin{landscape}
\subsection{Not controlling for teen height, pooled}
\input{../stats/output/regressions/OLShealth_noth_hq_pooled.tex}
\end{landscape}


\iffalse
\input{../stats/output/OLShealthcontrols_men.tex}

\input{../stats/output/OLShealthcontrols_women.tex}
\fi

\begin{appendix}
	\listoffigures
	\listoftables	
\end{appendix}


\end{document}